\documentclass{article}
\usepackage{amsmath}
\usepackage{amssymb}
\usepackage{graphicx}
\usepackage{hyperref}

\title{Relatório: Regressão Logística}
\author{Nome do Autor}
\date{\today}

\begin{document}

\maketitle

\begin{abstract}
Este relatório explica a regressão logística de forma sintetizada e matematicamente rigorosa. Inclui a formulação matemática, a interpretação dos coeficientes e a avaliação do modelo.
\end{abstract}

\section{Introdução}
A regressão logística é uma técnica estatística usada para modelar a probabilidade de um evento binário. É amplamente utilizada em estatística e aprendizado de máquina para classificação binária.

\section{Formulação Matemática}
A regressão logística modela a probabilidade de um evento \(Y\) ocorrer como uma função linear das variáveis preditoras \(X\).

\subsection{Função Sigmoide}
A função sigmoide é utilizada para mapear os valores da combinação linear para o intervalo \([0, 1]\):
\[
\sigma(z) = \frac{1}{1 + e^{-z}}
\]

\subsection{Modelo de Regressão Logística}
O modelo de regressão logística pode ser escrito como:
\[
P(Y=1|X) = \sigma(\beta_0 + \beta_1 X_1 + \beta_2 X_2 + \cdots + \beta_p X_p)
\]
onde \( \beta_0, \beta_1, \ldots, \beta_p \) são os coeficientes do modelo.

\subsection{Logito}
O logito transforma a probabilidade em uma combinação linear das variáveis preditoras:
\[
\text{logit}(P) = \log\left(\frac{P}{1-P}\right) = \beta_0 + \beta_1 X_1 + \beta_2 X_2 + \cdots + \beta_p X_p
\]

\section{Interpretação dos Coeficientes}
Os coeficientes da regressão logística representam a mudança no logito da probabilidade de \(Y\) para uma unidade de mudança na variável preditora.

\section{Estimativa dos Parâmetros}
Os parâmetros do modelo são estimados usando o método de máxima verossimilhança. A função de verossimilhança para \(n\) observações é:
\[
L(\beta) = \prod_{i=1}^n P(Y_i|X_i)^{Y_i} (1 - P(Y_i|X_i))^{1 - Y_i}
\]

\section{Avaliação do Modelo}
A qualidade do modelo pode ser avaliada usando diversas métricas, como a AUC-ROC, precisão, revocação e F1-score.

\section{Conclusão}
A regressão logística é uma ferramenta poderosa para a classificação binária. Compreender a formulação matemática e a interpretação dos coeficientes é crucial para a aplicação eficaz deste método.

\begin{thebibliography}{9}
\bibitem{book1}
Trevor Hastie, Robert Tibshirani, Jerome Friedman. \textit{The Elements of Statistical Learning}. Springer, 2009.
\end{thebibliography}

\end{document}
